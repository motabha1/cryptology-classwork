\documentclass[14pt]{article}
\usepackage{fullpage,enumitem,amsmath,amssymb,titlesec, xcolor, amsthm}
\titleformat*{\section}{\Large\bfseries}
\titlespacing*{\section}{-2em}{*2}{*0}
\titleformat*{\subsection}{\bfseries}
\usepackage{indentfirst}
\setlength\parindent{24pt}

\newtheorem{defn}{Definition}[section]
\newtheorem{coro}{Corollary}
\newtheorem{theorem}{Theorem}

\DeclareRobustCommand{\bbone}{\text{\usefont{U}{bbold}{m}{n}1}}

\DeclareMathOperator{\EX}{\mathbb{E}}

\begin{document}
	\title{\Huge \textbf{Lecture - 2}} 
	\author{Rachit Parikh}
	\date{February 24, 2022}
	\maketitle
	
	\noindent
	\rule{\linewidth}{0.4pt}
	
	\section{Cryptography}
		\noindent
		\linebreak
		What are the uses of cryptography?
		\begin{itemize}
			\item Digital Signatures
			\item Secure communication
			\item Integrity
		\end{itemize}
	
	\section{Encryption}
		\noindent
		\linebreak
		$\mathbb{K}$ - Key space\\
		$\mathbb{M}$ - Message space\\
		$\mathbb{C}$ - Ciphertext space\\
		$Enc(k, m)$ - Encryption function\\
		$Dec(k, c)$ - Decryption function\\
		
		\subsection{Correctness Property}
			\noindent
			$Dec(k, Enc(k, m_0)) = m_0$
			
		
		\subsection{Perfect Secrecy}
			\noindent
			If Bob and Alice are communicating and Eve is eavesdropping, then we have to ensure that she cannot guess the message. Since she can always guess out of message space, we have to make it the worst case. Basically she should not be able to do better than guessing randomly.\\
			
			
			\noindent
			First of all we begin with some notation before going to mathematical representation of definition of perfect secrecy.\\
			\linebreak
			$Eve(Enc(k, m_0)) = m_0 \rightarrow$ It means that Eve is able to guess the message correctly by looking at the encryption. This is sort of an event or an indicator random variable.\\
			\newpage
			$$X = [Eve(Enc(k, m_0)) = m_b], \text{ where}$$
			\linebreak
			$X = 1,$ if $b = 0$\\
			$X = 0,$ if $b \neq 0$\\
			\linebreak
			We can now formalize definition of perfect secrecy,\\
			\begin{defn}[Perfect Secrecy]
				\label{def}
				\item $m_0 \xleftarrow{R} \mathbb{M}\quad \implies \quad \Pr [X = 1] \leq \dfrac{1}{| \mathbb{M} |}$\\
				\linebreak
				Here, $m_0$ is selected uniformly from the message space $\mathbb{M}$. By this assumption, every message is equally likely so the best Eve can do is choose a message randomly and hence the above equation should hold if we want our encryption scheme to be \underline{perfectly secret}.\\
			\end{defn}
			\begin{coro}
				\label{coro}
				\item $b \xleftarrow{R} \{0, 1\} \quad \implies \quad \Pr [Eve(Enc(m_b, k)) = m_b] \leq \dfrac{1}{2}$
			\end{coro}
			\noindent
			Here, Eve cannot predict $m_0/m_1$ by seeing $Enc(m_1, k)$ or $Enc(m_0, k)$\\
			\linebreak
			Since the definition$\implies$corollary, we can also check whether the converse is true. Surprisingly, it is true and can be proven by considering contradiction. To prove Corollary$\implies$Definition, we can just prove that $\neg$Definition$\implies$$\neg$Corollary. 
			
			\begin{theorem} 
				Corollary 1$\implies$Definition 2.1
			\end{theorem}
			\begin{proof}
				Since we are proving it by contradiction, we just need to show that $\neg$Definition$\implies$$\neg$Corollary. First, we assume $\neg$Definition.\\
				$$\Pr [Eve(Enc(m_1, k)) = m_1] > \dfrac{1}{|\mathbb{M}|}$$
				Here $m_1$ is chosen randomly from the message space. Since this is an indicator random variable, we have
				$$\EX[Eve(Enc(m_1, k)) = m_1] > \dfrac{1}{|\mathbb{M}|}$$
				If we fix $m_0$, and our only randomness is $k \in \mathbb{K}$, then the probability that $m' = m_1$ can be given as 
				$$\Pr [Eve(Enc(m_0, k)) = m'] \leq \dfrac{1}{|\mathbb{M}|}$$
				This happens because $m_1$ is randomly chosen from the message space and the probability of it to be equal to $m'$ is at most $\dfrac{1}{|\mathbb{M}|}$
				$$\therefore \EX [Eve(Enc(m_0, k)) = m_1] \leq \dfrac{1}{|\mathbb{M}|}$$
				Now, if we take the difference of the random variables,
				
				$$\EX[Eve(Enc(m_1, k)) = m_1] - \EX [Eve(Enc(m_0, k)) = m_1] > 0$$
				This means that the probability of Eve guessing a random $m_1$ given $m_1$ is higher than Eve guessing it from some $m_0$. Hence, for some $m_1$, we have
				$$\Pr[Eve(Enc(m_1, k)) = m_1] > \Pr [Eve(Enc(m_0, k)) = m_1]$$
				Hence in the message space $b \xleftarrow{R} \{0, 1\}$,
				$$\Pr[Eve(Enc(m_b, k)) = m_b] > \dfrac{1}{2}$$
				Hence, we have proved $\neg$Corollary 1.
				
			\end{proof}
 		
\end{document}