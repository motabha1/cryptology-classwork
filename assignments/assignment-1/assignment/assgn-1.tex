\documentclass[14pt]{article}
\usepackage{fullpage,enumitem,amsmath,amssymb,titlesec, xcolor, hyperref}
\titleformat*{\section}{\large\bfseries}

\begin{document}
	
	\title{\color{blue}\Huge \textbf{Assignment-1}} 
	\date{\Large \textbf{Cryptology}}
	\author{Rachit Parikh (CRS-2101)}
	
	\maketitle
	\textbf{Disclaimer : }I declare that all the work presented in this assignment is my own work and I have only consulted the internet when it was absolutely necessary. 
	
	\noindent
	\rule{\linewidth}{0.4pt}
	
	\section*{Q-1 : Watch "The Imitation Game"}
		\noindent
		I have already seen this movie twice.\\
		
	\section*{Q-2 : Write C program to find primitive polynomials over $GF(2)$ for $n = \{4, \dots ,16\}$}
		\noindent
		\href{https://github.com/motabha1/cryptology-classwork/tree/main/assignments/assignment-1/code/primitive-polynomials}{\textcolor{blue}{Click here}} to get the code.\\
	\section*{Q-3 : Write C program to implement Stream Cipher}
		\noindent
		\href{https://github.com/motabha1/cryptology-classwork/tree/main/assignments/assignment-1/code/stream-cipher}{\textcolor{blue}{Click here}} to get the code.\\
	\section*{Q-4 : Understand Berlekamp-Massey algorithm}
		\noindent
		Algorithm to find linear complexity of a finite sequence and feedback polynomial of LFSR of minimal length which generates this sequence. It also states an important result that the LFSR of length $L$ which generates the sequence is unique iff $n \geq 2L$. I have also read about the algorithm and its proof from \href{https://www-users.cse.umn.edu/~garrett/students/reu/MB_algorithm.pdf}{\textcolor{blue}{here}}.
	\section*{Q-5 : Understand more about Non-linear feedback shift registers}
		\noindent
		The feedback bit is computed from a non-linear function of previous bits. An algorithm is known to generate the shortest (NL)FSR in linear time. Its application is in COS(vd) ciphers.
	
\end{document}
